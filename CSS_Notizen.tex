\documentclass[a4paper,12pt]{article}
\usepackage[utf8]{inputenc}
\usepackage[ngerman]{babel}

\usepackage{amsmath}
\usepackage{amsthm}
\usepackage{amsfonts}
\usepackage{amssymb}
\usepackage[left=3cm,right=2cm,top=2cm,bottom=1.5cm]{geometry}
\usepackage{parskip}
\usepackage{graphicx}
\usepackage{caption}
\usepackage{hyperref}
\usepackage{upgreek}
\hypersetup{
    colorlinks,
    citecolor=black,
    filecolor=black,
    linkcolor=red,
    urlcolor=red
}

%%Code style
\usepackage{listings}
\usepackage{xcolor}
\definecolor{codegreen}{rgb}{0,0.6,0}
\definecolor{codegray}{rgb}{0.5,0.5,0.5}
\definecolor{codepurple}{rgb}{0.58,0,0.82}
\definecolor{backcolour}{rgb}{0.9,0.9,0.9}

\lstset{escapeinside={@}{@}}
\lstdefinestyle{mystyle}{
    backgroundcolor=\color{backcolour},   
    commentstyle=\color{codegreen},
    keywordstyle=\color{magenta},
    numberstyle=\tiny\color{codegray},
    stringstyle=\color{codepurple},
    basicstyle=\ttfamily\footnotesize,
    breakatwhitespace=false,         
    breaklines=true,                 
    captionpos=b,                    
    keepspaces=true,                 
    numbers=left,                    
    numbersep=5pt,                  
    showspaces=false,                
    showstringspaces=false,
    showtabs=false,                  
    tabsize=5
}

\lstset{style=mystyle}

\newcommand{\blue}[1]{\textcolor{blue}{#1}}
\newcommand{\orange}[1]{\textcolor{orange}{#1}}
%%End code style

\title{Computer System Sicherheit\\Notizen WS20/21}
\author{Felix Marx}

\begin{document}
\maketitle

{
%%Lokales Einfärben des Inhaltverzeichisses
\hypersetup{linkcolor=black}
\tableofcontents
}
\newpage

\section{Begriffe}
Unterscheidung \blue{Betriebssicherheit (safety)} und \blue{Angriffssicherheit (security)}.\\
safety bezeichnet den Schutz vor inneren Fehlern, deren Eintrittswahrscheinlichkeit durch probabilistische Techniken ermittelt wird.\\
security den Schutz gegen aktive Angreifer.

\subsection{Security}

Trends und Herausforderungen:\\
\textbf{Mobilität}, der Übergang zu ''smart devices''
\begin{figure}[h!]
\centering
\includegraphics[scale=0.6]{Grafiken/Trend-Mobilitaet.png}
\caption{Herausforderungen bei der Mobilität}
\end{figure}

\textbf{Vernetzung}, der Übergang zu ''always on'' und Automatisierung von Systemen
\begin{figure}[h!]
\centering
\includegraphics[scale=0.6]{Grafiken/Trend-Vernetzung.png}
\caption{Herausforderungen bei der Vernetzung}
\end{figure}

\textbf{Miniaturisierung}, Produktion kleinerer Chips mit Trend zu Einwegchips
\begin{figure}[h!]
\centering
\includegraphics[scale=0.6]{Grafiken/Trend-Miniaturisierung.png}
\caption{Herausforderungen bei der Miniaturisierung}
\end{figure}\\

\blue{Social Engineering} bezeichnet das Erlangen von vertraulichen Daten durch psychologische Techniken\\
\blue{Man-in-the-middle} Angriffe, hier werden Nachrichten vor der gesicherten Kommunikation abgefangen.\\
Lösungen für \blue{Phishing} umfassen die Verifikation durch Transaktionsnummern (TANs) oder Hardwaretokens was allerding nicht die Integrität der Transaktion gewährleistet.\\
Schutzmechanismen:\\
\blue{Passive Authentication}: Sicherstellung der Authentizität durch Einsatz einer digitalen Signatur\\
\blue{Basic Access Control}: elektronische gespeicherte Daten werden erst übermittelt wenn das Lesegerät einen aufgedruckten maschinenlesbaren Code kennt.\\
\blue{Active Access Control}: Verhindert 1:1 Kopien indem ein geheimer Schlüssel in einem gesicherten Chip-Bereich gespeichert wird.\\
\blue{Extended Access Control}: Schutz von sensiblen Daten (Fingerabdruck, Iris, etc.) durch das Abgleichen von \href{https://en.wikipedia.org/wiki/Extended_Access_Control}{Zertifikaten} die nur eine kurze Zeit gültig sind.

\subsection{Safety}
Im Fehlerfall sollte ein System immer in den sicheren Zustand übergehen.\\
Zum Beispiel zeigt die Fahranweisung von Zügen nach oben, sodass es bei einem Schaden am Mechanismus in den Haltezustand nach unten fällt (Schwerkraft). Ein Watchdog (hier Schwerkraft) überwacht quasi das System auf Fehler und resettet es bei einem Fehler.

\section{Verlässliche Systeme}
Ein zuverlässiges System erfüllt seinen Zweck auch falls Fehler auftreten.
\begin{figure}
\centering
\includegraphics[scale=0.4]{Grafiken/AblaufZumVersagen.png}
\caption{Ablauf bis ein System versagt}
\label{Image:Versagensplan}
\end{figure}
Als Beispiel des \hyperref[Image:Versagensplan]{Ablaufes} lässt sich die Einwirkung von kosmischer Strahlung auf eine DRAM-Zelle (Fault) was zum Wechsel eines Wertes führt (Error), wodurch die Berechnung ein falsches Ergebnis liefert (Failure).\\
In der Zuverlässigkeit wird unterschieden nach:\\
\textbf{Availability}: Verfügbarkeit eines Systems gemessen in Prozent, d.h. 
$$\frac{\textrm{Total Up Time}}{\textrm{Total (Up + Down Time)}}=\frac{\textrm{MTTF}}{\textrm{MTTF + MTTR}}$$
\textbf{Reliability}: Zuverlässigkeit eines Systems, d.h. die Wahrscheinlichkeit dass ein System über einen gewissen Zeitraum korrekt funktioniert.\\
Wir bezeichnen mit \blue{MTTF} die Mean Time To Failure und mit \blue{MTTR} Mean Time To Recovery. Bei der Berechnung wird für die Up/Downtime nur die Zeiten im vereinbarten Betriebszeitraum gezählt!\\
Die Wahrscheinlichkeit, dass ein System bis zum Zeitpunkt $t$ fehlerhaft wird lässt sich mittels einer Verteilerfunktion $F$ und der Lebenszeit des Systems T berechnen:
$$F(t)=P(T\leq t)\textrm{, } R(t)=P(T>t)=1-F(t)$$
Die Funktion $R$ hingegen ermittelt die Wahrscheinlichkeit, dass ein System bis zum Zeitpunkt $t$ korrekt funktioniert.
Geht man davon aus, dass zukünftige Ausfälle unabhängig davon passieren, wann der letzte Ausfall war, lässt sich die Wahrscheinlichkeit mit einer Fehlerrate $\lambda$ so ausdrücken:\\
\begin{align*}
f(x) = &
\begin{cases} 
	\lambda e^{-\lambda x} & x\geq 0\\
	 0 & x < 0
\end{cases} \\
F(t) = & \int_{-\infty}^t f(x)dx = &
\begin{cases}
1 -e^{-\lambda t} & t \geq 0\\
0 & t < 0
\end{cases}\\
R(t) = & 1 -F(t) = & 
\begin{cases}
e^{-\lambda t} & t \geq 0\\
1 & t< 0
\end{cases}
\end{align*}
\begin{figure}[h!]
\centering
\includegraphics[scale=0.6]{Grafiken/MTTF-Model.png}
\caption{Berechnung MTTF mit dem exponentiellen Model}
\end{figure}\\
Bei in Reihe geschalteten Komponenten wird die Wahrscheinlichkeit dass das gesamte System korrekt funktioniert R durch $R_{ges}(t)=\prod^n_{i=1}R_i(t)$ im exponentiellen Modell wird dann für $\lambda = \lambda_{ges}=\sum_{i=1}^n\lambda_i$ verwendet.\\
Bei Parallelschaltung ist die Wahrscheinlichkeit dass das System korrekt funktioniert:
$$R_{ges}(t)=R_1(t)+R_2(t)-R_1(t)\cdot R_2(t), F_{ges}(t)=F_1(t)\cdot F_2(t)$$
Bei mehr als zwei Komponenten ist die allgemeine Formel:
$$R_{ges}(t)=1-\prod_{i=1}^n(1-R_i(t)), F_{ges}(t)=\prod_{i=1}^nF_i(t)$$

\subsection{Strategien zur Fehlervermeidung/toleranz}
\begin{itemize}
\item Fehlervermeidung (fault avoidance)
	\begin{itemize}
	\item Design des Systems stellt sicher, dass Fehler nicht auftreten
	\item Beispiel: Testen, Verifikation,...
	\end{itemize}
\item Wiederherstellung aus Fehlerzustand (fault recovery)
	\begin{itemize}
	\item Strategien, um ein System beim Auftreten eines Fehlers wieder in einen korrekten Systemzustand zu bringen
	\end{itemize}
\item Fehlertoleranz (fault tolerance)
	\begin{itemize}
	\item Wenn Fehler nicht vermieden werden können, dann soll das System Fehler tolerieren
	\item Beispiele: Redundanz, Safety, ...
	\end{itemize}
\end{itemize}
\textbf{Arten von Redundanzen:}
\begin{itemize}
\item Physikalische Redundanz (physical redundancy)
	\begin{itemize}
	\item Zusätzliche Ressourcen bzw. Komponenten
	\item Berechnungen werden auf mehreren Komponenten ausgeführt und verglichen
	\item Statische Redundanz
		\begin{itemize}
		\item $N$ Systeme laufen parallel
		\item $N-1$ Systeme laufen im  ''stand-by-mode''
		\item Bei einem Fehler wird im Betrieb umgeschaltet
		\item Dazu muss man den Fehler natürlich zuerst erkennen!
		\end{itemize}
	\item Dynamische Redundanz
		\begin{itemize}
		\item $N$ Systeme laufen parallel
		\item Ausfallsicherhere Komponente vergleicht die Resultate
		\item Mehrheitsentscheidung (zB. 2-out-of-3)
		\item Erkennt Fehler ''automatisch''
		\end{itemize}
	\end{itemize}
\item Zeitliche Redundanz (temporal redundancy)
	\begin{itemize}
	\item Berechnungen auf gleicher Hardware-Plattform wiederholen
	\end{itemize}
\item Redundanzen durch (Zusatz-)Information (information redundancy)
	\begin{itemize}
	\item Hinzufügen von zusätzlichen Daten (Checksummen, ...)
	\item Erlaubt Datenfelder bei Übertragung oder Speicherung zu erkennen
	\end{itemize}
\end{itemize}
Bei dynamischer Redundanz wird das Ergebnis parallel berechnet und dann per Mehrheitsentscheidung das korrekte ausgewählt. Problematisch falls im Vergleich ein Problem auftritt. Softwarefehler sind von der Hardware-Redundanz nicht abgedeckt.
Zusatzinformationen können die Integrität gesendeter Daten durch Checksummen sicherstellen, allerdings auch nur begrenzt.\\

Nachteile von Redundanzen sind:
\begin{itemize}
\item Schlechtere Performanz (bei temporaler Redundanz)
\item Synchronisation erforderlich
\item Hohe Kosten durch mehrfache Hardware bzw. mehrfache Implemenation (falls überhaupt möglich)
\item Benötigt Mechanismen zur Fehler-Erkennung, welche selbst wieder Fehleranfällig sein können
\end{itemize}

\section{Krypographie}
\begin{itemize}
\item Vertraulichkeit (Confidentiality)
	\begin{itemize}
	\item Schutz vor unbefugten Zugriff auf Informationen / Daten
	\end{itemize}
\item Integrität
	\begin{itemize}
	\item Schutz vor Veränderung von Informationen / Daten
	\end{itemize}
\item Verfügbarkeit (Availability)
	\begin{itemize}
	\item Daten oder Systeme sind verfügbar oder erreichbar
	\end{itemize}
\item Autehntizität (Authenticity)
	\begin{itemize}
	\item Schutz vor Fälschung von Informationen / Daten
	\end{itemize}
\item Nicht-Abstreitbarkeit (Non-Repudiation)
	\begin{itemize}
	\item Aktion ist nachprüfbar und kann nicht abgestritten werden
	\end{itemize}
\end{itemize}
\subsection{Begriffe}
\begin{itemize}
\item Klartext/Nachricht: eine zu verschlüsselnde Information
\item Klartextraum: Menge aller möglichen Klartexte
\item Chiffrat, Chiffretext: Verschlüsselte Nachricht
\item Chiffretextraum: Menge aller möglichen Chiffrate
\item Schlüssel: Ein Geheimnis, welches zur Ent-/Verschlüsselung benötigt wird
\item Schlüsselraum: Menge aller möglichen Schlüssel
\item Verschlüsselung: Umwandlung eines Klartext in ein Chiffrat
\item Entschlüsselung: Umwandlung eines Chiffrats in einen Klartext
\end{itemize}
\textbf{Kerckhoffs Prinzipien}
\begin{enumerate}
\item Das System muss praktisch, wenn nicht sogar mathematisch, unentschlüsselbar sein (''Unentschlüsselbarkeit'')
\item Es darf keine Geheimhaltung erfordern und darf ohne Schwierigkeiten  in die Hände des Feindes fallen (''Keine Geheimhaltung des Systems'')
\item Der Schlüssel muss ohne Hilfe geschriebener Notizen kommunizierbar und aufbewahrbar sein und er muss ausgewechselt oder modifiziert werden können nach Belieben der Kommunikationspartner (''Schlüssel ohne Aufschreiben'')
\item Es muss anwendbar sein auf die telegraphische Kommunikation
\item Es soll protabel sein und seine Funktion soll nicht die Zusammenkunft mehrerer Personen erfordern 
\item Schließlich ist es notwendig, angesichts der Umstände, unter denen es angewendet werden soll, dass das System einfach benutzbar ist und weder große gedankliche Anstrengung erfordert noch die Kenntnis einer langen Liste zu beachtender Regeln (''einfach benutzbar'')
\end{enumerate}
\subsection{Verschiebechiffre}
Die Ceasar-Chiffre ersetzt jeden Buchstaben des Klartextes durch den Buchstaben 3 Stellen weiter rechts, beim Entschlüsseln genau umgekehrt.\\
Die Verschiebechiffre ersetzt alle Buchstaben durch den Buchstaben welcher eine beliebige aber feste Anzahl an Stellen weiter rechts steht. Der Schlüssel ist die Anzahl an Verschiebungen, d.h. 0,1,2,...,25.

\subsection{Mathematische Grundlagen}
Als \blue{Einwegfunktionen} bezeichnet man Funktionen bei denen es ''einfach'' ist den Funktionswert $y = f(x)\in Y$ zu bestimmen, aber ''schwer'' ein Urbild $x\in f^{-1}[\{y\}]$ zu finden. Die Existenz von Einwegfunktionen ist nicht bewiesen (P/NP).
\subsubsection{Teilbarkeitsregeln}
\begin{align}
a | a\\
a | b \wedge b |c \Rightarrow a|c\\
a|b\wedge b\neq 0 \Rightarrow |a|\leq |b|\\
a | b\wedge b|a \Rightarrow |a| = |b|
\end{align}
Sind $a,n\in\mathbb{Z}$ mit $n\neq 0$, dann gibt es eindeutig bestimmte Zahlen $q,r\in\mathbb{Z}$ sodass
\begin{align*}
a= qn+r\\
0\leq r < |n|\\
q = \lfloor \frac{a}{n} \rfloor \wedge r = a -qn
\end{align*}
\subsubsection{Größter gemeinsamer Teiler}
\begin{align}
\textrm{ggT}(a,0)=|a|\\
\textrm{ggT}(a,1)=1\\
\textrm{ggT}(a,b)=\textrm{ggT}(b,a)\\
\textrm{ggT}(a,b)=\textrm{ggT}(|a|,|b|)\\
\textrm{ggT}(a,b)=\textrm{ggT}(b,a-b)\\
\textrm{Für } b\neq 0 \textrm{ gilt } \textrm{ggT}(a,b)=\textrm{ggT}(b,a \textrm{ mod } b)
\end{align}

Lineare diophantische Gleichung: $ax+by=c$\\
Eine diophantische Gleichung kann wie folgt gelöst werden.\\
Löse $ax'+by'=$ggT(a,b) und definiere $x=\frac{c}{\textrm{ggT(a,b)}}\cdot x'$, $y=\frac{c}{\textrm{ggT(a,b)}}\cdot y'$

\textbf{Beispiel erweiteter Euklid:}
$1337\cdot x+42\cdot y=\blue{7}$\\
\begin{tabular}{|l|l|l|l|l|}
a & b & $\lfloor \frac{a}{b} \rfloor$ & x & y\\
\hline
1337 & 42 & 31 & -1 & $1-31*(-1)=32$\\
\hline
42 & 35 & 1 & 1 & $0-1*1=-1$\\
\hline
35 & 7 & 5 & 0 & 1\\
\hline
\blue{7} & 0 & & 1 & 0
\end{tabular}\\
\subsubsection{Primzahlen}
Fundamentalsatz der Arithmetik:\\
Jede natürliche Zahl $n>1$ besitzt eine Zerlegung in ein Produkt aus Primzahlen, welche bis auf die Reihenfolge eindeutig ist.\\
Es gibt unendlich viele Primzahlen.\\

Die Eulersche-Phi-Funktion $$\varphi (b)=|\{a\in\mathbb{N} : a < b, \textrm{ggT}(a,b)=1\}|$$ beschreibt dei Anzahl der zu b teilerfremden Zahlen.\\
Für eine Primzahl p gilt $\varphi (p)=p-1$ und $\varphi (p^n)=p^n-p^{n-1}$.\\
Für teilerfremde Zahlen $a,b\in \mathbb{N}$ gilt $\varphi (ab)=\varphi (a)\cdot \varphi (b)$.\\
Sind $m,n\in \mathbb{N}$ teilerfremd, dann gilt $m^{\varphi (n)}\textrm{mod }n= 1$.\\
\textbf{Kleiner Satz von Fermat:}\\
Für eine Primzahl p und ein teilerfremde Zahl $m\in \mathbb{N}$ gilt $m^{p-1} \textrm{mod }p = 1$.
\subsection{Symmetrische Kryptosysteme}
In symmetrischen Kryptosystemen ist der Schlüssel zum Ver- und Entschlüsseln der selbe.\\
Formal bezeichnet ein symmetrisches Kryptosystem ein 5-Tupel $(\mathcal{M},\mathcal{K}, \mathcal{C},e,d)$ mit $\mathcal{M}$ als Menge an Klartexten, $\mathcal{K}$ als Menge von Schlüsseln, $\mathcal{C}$ als Menge an Chiffretexten.
$e: \mathcal{M}\times\mathcal{K}\rightarrow\mathcal{C}$ ist die Verschlüsselungsfunktion und $d: \mathcal{C}\times\mathcal{K}\rightarrow\mathcal{M}$ als Entschlüsselungsfunktion.\\
Weiter werden folgende Funktionen eingeführt:
\begin{align*}
num: & \{ A,B,C,...,Z\}\rightarrow\{ 0,1,2,...,25\} \textrm{ Zuordnung der Zahlenwerte}\\
char: & \{ 0,1,2,...,25\}\rightarrow\{ A,B,C,...,Z\}\\
sr: & \{A,...,Z\}\times\{0,1,...,25\}\rightarrow\{A,...,Z\} \textrm{ Ist ein Rechtsshift}\\
sl: & \{A,...,Z\}\times\{0,1,...,25\}\rightarrow\{A,...,Z\} \textrm{ Linksshift}
\end{align*}
Damit ergibt sich die für Verschiebechiffren die folgenden Interpretationen:
\begin{align*}
e(w_0w_1...w_n,k) &= sr(w_0,k)sr(w_1,k)...sr(w_n,k)\\
d(w_0w_1...w_n,k) &= sl(w_0,k)sl(w_1,k)...sl(w_n,k)
\end{align*}
Monoalphabetische Chiffretexte können durch Häufigkeitsanalysen gebrochen werden.
Um das zu verhinden existiert das \blue{One Time Pad} hierbei hat der Schlüssel die selbe Länge wie der Klartext, d.h. jeder Buchstabe im Klartext wird mit einem eigenen Schlüsselbuchstaben verschlüsselt. Das Kryptosystem sieht dann so aus: $(\mathcal{M}^n,\mathcal{K}^n, \mathcal{C}^n,e,d)$
Die Bitweise veroderung ist ein One Time Pad $(\mathbb{Z}_2^n,\mathbb{Z}_2^n,\mathbb{Z}_2^n,e,d)$ mit 
\begin{align*}
e:\mathbb{Z}_2^n\times\mathbb{Z}_2^n\rightarrow\mathbb{Z}_2^n,e(x,k)=x\oplus k\\
d:\mathbb{Z}_2^n\times\mathbb{Z}_2^n\rightarrow\mathbb{Z}_2^n,d(y,k)=y\oplus k
\end{align*}
Ein Kryptosystem heißt \blue{perfekt sicher} wenn für einen gegebenen Chiffretext jeder Klartext gleich wahrscheinlich ist, d.h. das One Time Pad ist bei gleichverteiltem zufälligen Schlüssel perfekt sicher.
Ein Kryptosystem heißt \blue{semantisch sicher}, wenn es für einen Angreifer der die Länge der Nachricht und das Chiffrat kennt nicht wesentlich einfacher ist auf den Klartext zu schließen als für einen Angreifer der nur die Länge des Klartextes kennt.\\
Als \blue{Blockchiffren} werden Kryptosysteme bezeichnet deren Klartexte $M^n$ und Chiffretexte $C^n$ alle eine Blocklänge n und deren Schlüssel $K^m$ die Schlüssellänge m haben.
Blockchiffre Verfahren sind unter anderem:
\begin{itemize}
\item Advanced Enrcyption Standard (AES) oder Rijndael
	\begin{itemize}
	\item Blocklänge: $n= 128$
	\item Schlüssellänge: $m\in \{128,192,256\}$
	\end{itemize}
\item Data Encryption Standard (DES)
	\begin{itemize}
	\item Gebrochen für $n=64$ und $m=56$
	\item 3DES mit $n=64$ und $m=168$ hat die selbe Sicherheit wie 112 Bit Schlüssel
	\end{itemize}
\item Serpent ($n=128$, $m\in\{128,192,256\}$)
\item Twofish ($n=128$, $m\in\{128,192,256\}$)
\item Blowfish ($n=64$, $32\leq m\leq 448$ - Standard: $m=128$)
\end{itemize}
\subsubsection{Vigenère mit alphabetischen Schlüssel, Länge n}
Beim Vignerère wird der Klartext entsprechend eines Codewortes verschlüsselt. Dabei wird der aktuelle Buchstabe des Klartextes um die Wertigkeit des aktuellen Buchstabens des Codewortes im Alphabet nach rechts verschoben. Ist man am Ende des Schlüsselwortes angekommen beginnt man wieder am ersten Buchstaben.\\
Verwendet man einen zufällig erstellten Schlüssel, der genauso lang ist wie der Klartext und benutzt ihn nur ein einziges Mal, so ist die Verschlüsselung perfekt sicher, kann also ohne Schlüssel nicht entschlüsselt werden (One-Time-Pad).
Zur Darstellung der Verschiebung kann ein Vigenère-Quadrat verwendet werden. Der Schlüsselraum ist $25^n$.\\
\begin{figure}[h!]
\centering
\includegraphics[scale=0.7]{Grafiken/VigenereSquare.jpg}
\caption{Vigenère Quadradt, an der X-Achse ist die Verschiebung und Y-Achse der Klarbuchstabe}
\end{figure}
\subsubsection{Electronic Codebook Modus}
Im ECB Modus werden die einzelnen Blöcke im Blockchiffre Kryptosystem jeweils mit dem selben Schlüssel verschlüsselt und die einzelnen Chiffretexte kombiniert um den gesamten Chiffretext zu bekommen. Die \hyperref[pic:ECBModus]{Entschlüsselung} läuft analog.
Da nicht alle Blöcke die genau vom Kryptosystem geforderte Länge besitzen müssen die Blöcke teilweise mit einer Auffüllfunktion $pad: M^*\rightarrow (M^n)^*$ gefüllt idealerweise sollte eine pad Funktion wieder umkehrbar sein.
Der ECB Modus ist für Klartexte welche in mehrere Blocke augeteilt werden unsicher und sollte nie verwendet werden.\\
Im ECB Modus ist die Ver- und Entschlüsselung parallelisierbar.

\begin{figure}
\centering
\includegraphics[scale=0.5]{Grafiken/ECBModus.png}
\caption{Entschlüsselung im ECB Modus}
\label{pic:ECBModus}
\end{figure}

\subsubsection{Cipher Block Chaining Modus}
Der CBC Modus ist eine Verbesserung des ECB Modus, da hier für die Ver- und \hyperref[fig:CBCModus]{Entschlüsselung} jeweils der vorherige Block als Rauschen mit dem Klartext verodert wird, sodass auch gleiche Muster im Klartext verschiedene Ergebnisse im Chiffretext produzieren. Dafür erweitern wir das Kryptosystem um die Menge der Zufallswerte $\mathcal{R}$ aus welcher der Initialisierungvektor gewählt wird. Hierbei ist $\mathcal{R}$ nicht geheim und kann unverschlüsselt übertragen werden.\\
Es soll $e(x,k,r_1)\neq e(x,k,r_2)$ gelten. Für die Sicherheit der Verschlüsselung ist wichtig, dass jeder Zufallswert nur einmal verwendet wird.\\
Das erweiterte Kryptosystem heißt \blue{randomisiertes symmetrisches Kryptosystem} und wird als 6-Tupel dargestellt: $(\mathcal{M},\mathcal{K},\mathcal{C},\mathcal{C},\mathcal{R},e^*,d^*)$. Der Definitionsbereich von $e^*$ und $d^*$ wird um $\mathcal{R}$ erweitert.
$$y_i = e(x_i\oplus y_{i-1},k), x_i = d(y_i,k)\oplus y_{i-1} \textrm{ für } i\geq 1$$
\begin{figure}
\centering
\includegraphics[scale=0.5]{Grafiken/CBCModus.png}
\caption{Entschlüsselung im CBC Modus, bei der Entschlüsselung umgedreht, sodass Parallelisierung unmöglich ist}
\label{pic:CBCModus}
\end{figure}
Die Verschlüsselung ist damit nicht mehr parallelisierbar, aber die Entschlüsselung bleibt es.\\

Damit die Verschlüsselung auch parallelisierbar ist gibt es den \blue{Counter (CTR) Modus} hier wird bei der Verschlüsselung für jeden Block der Initialiserungsvektor mit einem Zählerwert verschlüsselt und erst dann mit dem Klartext verodert. Die \hyperref[pic:CTRModus]{Entschlüsselung} läuft genau gleich ab, nur das hier der Ciffretext verodert wird.\\
\begin{figure}
\centering
\includegraphics[scale=0.5]{Grafiken/CounterCTRModus.png}
\caption{Entschlüsselung im CTR Modus, Nonce ist der Initialiserungsvektor}
\label{pic:CTRModus}
\end{figure}
Die Wahl des Initialisierungsvektor muss unvorhersagbar sein, das heißt nicht das die Initialisierungsvektoren selber geheim bleiben müssen (die Sicherheit des Chiffretextes garantiert der Schlüssel) sondern dass die Wahl zukünftiger Initialisierungsvektoren nicht vorhersagbar sein dar (sie also von geheimen Informationen abhängen) und ein Angreifer die Wahl nicht beeinflussen kann\footnote{\url{https://tinyurl.com/y5x3bd5j} BSI Seite 76}.\\
Möglichkeiten zur Wahl des Initialisierungsvektor sind:
\begin{itemize}
\item Zufällige Initialisierungvektoren, d.h. eine zufällige Bitfolge der Länge n, hier muss die Bitfolge aber ein Entropie von 95 Bit besitzen ($n\geq95$?)
\item Verschlüsselte Initialisierungsvektoren, wir wählen einen determistisch erzeugten Wert und verschlüsseln ihn mit der einzusetzenden Blockchiffre und Schlüssel, der Chiffretext ist der Initialisierungsvektor
\end{itemize}

Der randomatisierte Zähler wird als eine Funktion $ctr: \mathbb{Z}_2^*\times \mathbb{N}_0\rightarrow \mathbb{Z}_2^n$  definiert welche die basierend auf dem gegebenen Zufallswert eine Erhöhung in unspezifierter Weise ausführt.\\
Mögliche Implementationen umfassen:
\begin{itemize}
\item Einfacher Zähler, welcher den Zufallswert um n erhöht
\item  Die Hälfte der Bits des Zufallswertes sind reserviert und nur die rechte Hälfte der Bits wird erhöht und läuft über\footnote{\url{https://tinyurl.com/y27ga7ew} NST Seite 18ff, Anhang B}
\end{itemize}
Das wiederverwenden des Zufallswertet kompromitiert die Sicherheit der Verschlüsselung, da sobald ein Ciffretext einen Klartext zugeordnet wurde der Schlüssel ermittelt werden kann.\\
Das CTR Modus Kryptosystem wird formal so bezeichnet $((\mathbb{Z}_2^n)^*,\mathbb{Z}_2^n,(\mathbb{Z}_2^n)^*,\mathbb{Z}_2^*,e^*,e^*)$ wobei e und d wie folgt definiert sind:
\begin{align*}
e^*:(\mathbb{Z}_2^n)^*\times \mathbb{Z}_2^m \times \mathbb{Z}_2^n\rightarrow (\mathbb{Z}_2^n)^*, e^*(x_0x_1...x_l,k,r)=y_0y_1...y_l \textrm{ mit } y_i = e(ctr(r,i),k)\oplus x_i\\
d^*:(\mathbb{Z}_2^n)^*\times \mathbb{Z}_2^m \times \mathbb{Z}_2^n\rightarrow (\mathbb{Z}_2^n)^*, d^*(y_0y_1...y_l,k,r)=x_0x_1...x_l \textrm{ mit } x_i = e(ctr(r,i),k)\oplus y_i\\
\end{align*}
\subsubsection{Stromchiffren}
Stromchiffren sind pseudozufällige Schlüsselströme welche aus dem Schlüsselwert generiert werden und zur Ver- und Entschlüsselung mittels xor mit den Klar-/Chiffretext kombiniert werden. Deshalb muss der Schlüssel sicher erzeugt sein, da die Sicherheit des Verfahrens nur davon abhängt, denn ein Pseudozufallszahlengenerator ist deterministisch.\\
Formal wird ein Stromchiffre Kryptosystem so bezeichnet: $(\mathbb{Z}_2^*,\mathbb{Z}_2^k,\mathbb{Z}_2^*,e,d)$ mit der Funktion:
\begin{align*}
keystream: \mathbb{Z}_2^*\times \mathbb{Z}_2^k\rightarrow\mathbb{Z}_2^* \textrm{ mit } |keystream(x,z)|=|x|\\
e(x,z)=d(x,z) = x\oplus keystream(x,z)
\end{align*}
Hierbei liegt es an der Implementation des keystreams ob dieser auch vom ersten Parameter abhängig ist oder ob die Zufallswert nur aus dem Schlüssel erzeugt werden.
\end{document}